\documentclass[a4paper, 12pt]{article}

\begin{document}

MAKERERE UNIVERSITY\\
COLLEGE OF COMPUTING AND INFORMATION SCIENCES\\
COMPUTER SCIENCE YEAR 2\\
RESEARCH METHEDOLOGY\\ \\

NABENDE EDWARD\\
\textmd{16/U/8095/PS}\\
\textmd{216012611}\\ \\

 \section{  RESEARCH ON UNEMPLOYMENT} 
\subsection{Defining unemployment}

Unemploymen is the state of being unemployed . OR It can be defined as  is a phenomenon that occurs when a person who is actively searching for employment is unable to find work.\\
Though with regard to economics, The unemployment rate is a measure of the prevalence of unemployment and it is calculated as a percentage by dividing the number of unemployed individuals by all individuals currently in the labor force.
\subsection{Why research on unemployment}
This is to help all the bothered government bodies to clearly make resolution  on how to reduce on  unemployment using  the collected data.
\subsection{Methods of data collection}
Interviews, questinnaires and surveys were used to get the information for tabulation.

\section{UNEMPLOYMENT FORM}


\begin{tabular}{|c|c|c|c|c|c|}
\hline
Name& DistrictName &Eductionstatus& Employed \\ [0.5ex]
\hline
Ojok & Apac & Low & no\\ [0.5ex]
\hline
Abila & Tororo & High & no \\ [0.5ex]
\hline
Mwaita & Mbarara & High& yes\\ [0.5ex]
\hline
Massiga & Busia & Medium&yes\\ [0.5ex]
\hline
SSentoogo & Masaka & Low & no\\ [0.5ex]
\hline
\end{tabular}


\end{document}